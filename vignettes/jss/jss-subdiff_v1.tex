\documentclass[article]{jss}

%% -- LaTeX packages and custom commands ---------------------------------------

%% recommended packages
\usepackage{thumbpdf,lmodern}

%% other packages
\usepackage{amsmath,amsfonts,amsthm}
% \usepackage{physics}
\usepackage{bm}
\usepackage{framed} % (only for this demo article)

%% new custom commands
\newcommand{\class}[1]{`\code{#1}'}
\newcommand{\fct}[1]{\code{#1()}}
% \newcommand{\bm}[1]{\boldsymbol{#1}}
\newcommand{\nd}{k}
\newcommand{\np}{d}
\newcommand{\aD}{(\alpha,D)}
\newcommand{\tth}{\bm{\theta}}
\newcommand{\pph}{\bm{\phi}}
\newcommand{\pps}{\bm{\psi}}
\newcommand{\aal}{\bm{\alpha}}
\newcommand{\bbe}{\bm{\beta}}
\newcommand{\gga}{\bm{\gamma}}
\newcommand{\FF}{\bm{F}}
\newcommand{\mmu}{\bm{\mu}}
\newcommand{\lla}{\bm{\lambda}}
\newcommand{\rrh}{\bm{\rho}}
\newcommand{\nnu}{\bm{\nu}}
\newcommand{\LLa}{\bm{\Lambda}}
\newcommand{\UUp}{\bm{\Upsilon}}
\newcommand{\SSi}{\bm{\Sigma}}
\newcommand{\OOm}{\bm{\Omega}}
\newcommand{\TTh}{\bm{\Theta}}
\newcommand{\PPs}{\bm{\Psi}}
\newcommand{\XX}{\bm{X}}
\newcommand{\JJ}{\bm{J}}
\newcommand{\dXX}{\Delta\XX}
\newcommand{\dt}{\Delta t}
\newcommand{\VV}{\bm{V}}
\newcommand{\ZZ}{\bm{Z}}
\newcommand{\dZZ}{\Delta\ZZ}
\newcommand{\UU}{\bm{U}}
\newcommand{\R}{\mathbb{R}}
\newcommand{\ind}{\stackrel{\mathrm{ind}}{\sim}}
\newcommand{\iid}{\stackrel{\mathrm{iid}}{\sim}}
\newcommand{\FI}{\bm{\mathcal{I}}}
\newcommand{\tmin}{t_{\textnormal{min}}}
\newcommand{\tmax}{t_{\textnormal{max}}}
\newcommand{\rv}[3][1]{#2_{#1},\ldots,#2_{#3}}
\newcommand{\ctrans}{\bm{g}}
% \newcommand{\logit}{\mathrm{logit}}
\DeclareMathOperator{\logit}{logit}
\DeclareMathOperator{\ilogit}{ilogit}
\DeclareMathOperator{\matnorm}{MatNorm}
\DeclareMathOperator{\mniw}{MNIW}
\DeclareMathOperator{\dir}{Dirichlet}
\DeclareMathOperator{\N}{\mathcal{N}}
\DeclareMathOperator{\toep}{Toepliz}
\DeclareMathOperator{\var}{var}
\DeclareMathOperator{\cov}{cov}
\DeclareMathOperator{\se}{se}
\DeclareMathOperator{\msd}{\textsc{msd}}
\DeclareMathOperator{\acf}{\textsc{acf}}
\DeclareMathOperator{\tr}{tr}
% \DeclareMathOperator{\logit}{logit}


%% -- Article metainformation (author, title, ...) -----------------------------

%% - \author{} with primary affiliation
%% - \Plainauthor{} without affiliations
%% - Separate authors by \And or \AND (in \author) or by comma (in \Plainauthor).
%% - \AND starts a new line, \And does not.
\author{
  Martin Lysy\\University of Waterloo
  \And
  Yun Ling\\University of Waterloo
  \AND
  Katherine Slyman\\University of North Carolina\\Chapel Hill
  \And
  Katherine Sutherlin\\University of North Carolina\\Chapel Hill
  \AND
  Neall Caughman\\University of North Carolina\\Chapel Hill
  \And
  M.~Gregory Forest\\University of Carolina\\Chapel Hill
}
\Plainauthor{Martin Lysy, Yun Ling}

%% - \title{} in title case
%% - \Plaintitle{} without LaTeX markup (if any)
%% - \Shorttitle{} with LaTeX markup (if any), used as running title
\title{\pkg{subdiff}: An \proglang{R} Toolchain for Analysis of Subdiffusion in Single-Particle Tracking}
\Plaintitle{subdiff: An R Toolchain for Analysis of Subdiffusion in Single-Particle Tracking}
\Shorttitle{The \pkg{subdiff} Toolchain for Single-Particle Tracking}

%% - \Abstract{} almost as usual
\Abstract{\noindent%
  \begin{itemize}
  \item Scientific importance of single-particle tracking and understanding subdiffusion.
  \item Challenges: cost of likelihood evaluations.  camera errors.  heterogeneity of fluids.
  \item Contribution: Fast and flexible set of tools to analysis particle tracking data.
  \end{itemize}
  % This short article illustrates how to write a manuscript for the
  % \emph{Journal of Statistical Software} (JSS) using its {\LaTeX} style files.
  % Generally, we ask to follow JSS's style guide and FAQs precisely. Also,
  % it is recommended to keep the {\LaTeX} code as simple as possible,
  % i.e., avoid inclusion of packages/commands that are not necessary.
  % For outlining the typical structure of a JSS article some brief text snippets
  % are employed that have been inspired by \cite{Zeileis+Kleiber+Jackman:2008},
  % discussing count data regression in \proglang{R}. Editorial comments and
  % instructions are marked by vertical bars.
}

%% - \Keywords{} with LaTeX markup, at least one required
%% - \Plainkeywords{} without LaTeX markup (if necessary)
%% - Should be comma-separated and in sentence case.
% \Keywords{JSS, style guide, comma-separated, not capitalized, \proglang{R}}
% \Plainkeywords{JSS, style guide, comma-separated, not capitalized, R}
\Keywords{Single-particle tracking, subdiffusion, etc.}

%% - \Address{} of at least one author
%% - May contain multiple affiliations for each author
%%   (in extra lines, separated by \emph{and}\\).
%% - May contain multiple authors for the same affiliation
%%   (in the same first line, separated by comma).
\Address{
  Martin Lysy\\
  Department of Statistics and Actuarial Science\\
  University of Waterloo\\
  E-mail: \email{mlysy@uwaterloo.ca}%\\
  % URL: \url{https://eeecon.uibk.ac.at/~zeileis/}
}

\begin{document}


%% -- Introduction -------------------------------------------------------------

\section{Introduction} \label{sec:intro}

%% -- Manuscript ---------------------------------------------------------------

\section{Subdiffusion models and estimators}

Follow AOAS pretty closely:
\begin{itemize}
\item Empirical MSD estimator
\item OLS subdiffusion estimator
\item Location-Scale parametric model
\end{itemize}

\subsection{Overview}

The \pkg{subdiff} package provides a number of utilities for analyzing subdiffusion in passive particle tracking data.  Let $\XX(t) = \big(X_1(t), \ldots, X_k(t)\big)$ denote the $k$-dimensional position of a particle at time $t$, where $k = 1-3$.  In an ideal experiment, it is typically assumed that $\XX(t)$ is a Continuous Stationary Increment (CSI) process, i.e., a stochastic process with mean zero and such that the distribution of $\dXX(t) = \XX(t+\dt) - \XX(t)$ is independent of $t$.  Under these conditions, the covariance properties of $\XX(t)$ can be deduced entirely from the particle's mean square displacement (MSD),
\begin{equation}\label{eq:msd}
  \msd_{\XX}(t) = \frac 1 \nd \times E\bigl[\lVert \XX(t) - \XX(0) \rVert^2\bigr] = \frac 1 \nd \times \sum_{j=1}^\nd E\bigl[\lvert X_j(t) - X_j(0) \rvert^2\bigr].
\end{equation}
A widely observed phenomenon is that the MSD of microparticles diffusing in biological fluids has a power-law signature on a given timescale,
\begin{equation}
  \msd_{\XX}(t) \sim D t^\alpha, \qquad t \in (\tmin, \tmax).
  \label{eq:subdiff}
\end{equation}
Whereas $\alpha = 1$ corresponds to the MSD of ordinary diffusion observed in Brownian particles, the power law MSD \eqref{eq:subdiff} with $\alpha \in (0,1)$ is referred to as subdiffusion.  

\subsection{Empirical MSD and Semiparametric Subdiffusion Estimator}

The \code{hbe} dataset included in \pkg{subdiff} package consists of \code{`r length(unique(hbe[["id"]]))`} 2D trajectories consisting of \code{`r n_rng[1]`-`r n_rng[2]`} observations recorded at a frequency of \code{`r 1/dT`}Hz.  These trajectories are displayed in Figure~\ref{fig:hbe_plot}. 

Let $\XX = (\rv [0] \XX N)$, $\XX_n = \XX(n \cdot \dt)$ denote recorded positions of a given particle trajectory.  The semi-parametric subdiffusion estimator consists of two steps:
\begin{enumerate}
\item Calculate the empirical MSD for each trajectory, defined as
  \[
    \widehat{\msd}_{\XX}(n \cdot \dt) = \frac 1 {q(N-n+1)} \sum_{h=0}^{N-n} \Vert \XX_{h+n}-\XX_{h}\Vert^2.
  \]
\item Estimate $\aD$ by regressing $y_n = \log \widehat{\msd}_{\XX}(n \cdot \dt)$ onto $x_n = \log(n\cdot \dt)$, such that
  \begin{equation}\label{eq:ols}
    \hat \alpha = \frac{\sum_{n=0}^N(y_n - \bar y)(x_n - \bar x)}{\sum_{n=0}^N(x_n - \bar x)^2}, \qquad \hat D = \tfrac 1 2 \exp(\bar y - \hat \alpha \bar x),
  \end{equation}
  where $\bar x$ and $\bar y$ are the samples means of $x_n$ and $y_n$.
\end{enumerate}
	
Since particle trajectories are often contaminated with low-frequency drift, the empirical MSD is often calculated on the drift-subtracted observations, $\tilde \XX_n = (\XX_n - \XX_0) - n \overline{\dXX}$, where $\dXX = \frac 1 N \sum_{n=1}^N (\XX_n - \XX_{n-1})$ is the average displacement over the interobservation time $\dt$.  Morever, the empirical MSD is known to be biased at large lag times, such that only about 30-50\% of the shortest lagtimes are usually kept for estimating $\aD$.

Figure~\ref{fig:msd_emp} displays the empirical MSD of the first 500 lags for each trajectory and the corresponding estimates of $\aD$.

\subsection{Fully-Parametric Subdiffusion Estimators}\label{sec:parest}

While the semiparametric estimator~\eqref{eq:ols} operates under minimal modeling assumptions, complete specification of the stochastic process $\XX(t)$ provides not only a considerable increase in statistical efficiency~\citep[e.g.,][]{mellnik.etal16}, but in fact is necessary to establish dynamical properties of particle-fluid interactions 
% -- such as first-passage times of microparticle pathogens through protective mucosal layers --
which cannot be determined from second-order moments (such as the MSD) alone~\citep{gal.etal13,lysy.etal16}.  A convenient framework for stochastic subdiffusion modeling is the location-scale model of~\cite{lysy.etal16},
\begin{equation}\label{eq:lsmodel}
  \XX(t) = \sum_{j=1}^\np \bbe_j f_j(t) + \SSi^{1/2} \ZZ(t),
\end{equation}
where $f_1(t), \ldots f_\np(t)$ are known functions accounting for low-frequency drift (typically linear, $f_1(t) = t$, and occasionally quadratic, $f_2(t) = t^2$), $\rv \bbe \np \in \mathbb R^\nd$ are regression coefficients, $\SSi_{\nd \times \nd}$ is a variance matrix, and $\ZZ(t) = \bigl(Z_1(t), \ldots, Z_\nd(t)\bigr)$ are iid continuous stationary-increments (CSI) Gaussian processes with mean zero and MSD parametrized by $\pph$,
\begin{equation}\label{eq:csimsd}
  \msd_Z(t) = E\bigl[\lVert Z_j(t) - Z_j(0) \rVert^2\bigr] = \eta(t \mid \pph),
\end{equation}
such that the MSD of the drift-subtracted process $\tilde \XX(t) = \XX(t) - \sum_{j=1}^\np \bbe_j f_j(t)$ is given by
\begin{equation}\label{eq:csimsddrift}
  \msd_{\tilde \XX}(t) = \tfrac{1}{\nd} \tr(\SSi) \cdot \eta(t \mid \pph).
\end{equation}

Perhaps the simplest parametric subdiffusion model sets $Z_j(t) = B_\alpha(t)$ to be fractional Brownian Motion (fBM)~\citep[e.g.,][]{szymanski.weiss09,weiss13}, a mean-zero CSI Gaussian process with covariance function
\begin{equation}\label{eq:fbm-cov}
  \cov\bigl(B_\alpha(t), B_\alpha(s)\bigr) = \tfrac 1 2 (|t|^\alpha + |s|^\alpha - |t-s|^\alpha), \qquad 0 < \alpha < 2.
\end{equation}
Indeed, as the covariance function of a CSI process is completely determined by its MSD, fBM is the only (mean-zero) CSI Gaussian process exhibiting \emph{uniform} subdiffusion,
\begin{equation}\label{eq:fbm-msd}
  \msd_{B_\alpha}(t) = t^\alpha, \qquad 0 < t < \infty,
  % \qquad 0 < \alpha < 1
\end{equation}
in which case the diffusivity coefficient is given by
\[
  D = \frac{1}{2\nd} \times \tr(\SSi).
\]
Other examples of driving CSI processes are the confined diffusion model of~\cite{ernst.etal17} and the viscoelastic Generalized Langevin Equation (GLE) of~\cite{mckinley.etal09}, both of which exhibit \emph{transient}
% (anomalous)
subdiffusion, i.e., power-law scaling only on a given timescale $t \in (\tmin, \tmax)$. In this case, the subdiffusion parameters $\aD$ become functions of the other parameters, namely $\alpha = \alpha(\pph)$ and $D = D(\pph, \SSi)$.  We shall revisit these transient subdiffusion models in Section~\ref{sec:sim}.


% Let $\XX = (\rv [0] \XX N)$, $\XX_n = \XX(n\cdot \dt)$, denote the discrete-time observations of a given particle trajectory $\XX(t)$ recorded at frequency $1/\dt$. 
% Assuming that %the position process
% $\XX(t)$ has second-order stationary increments,
% \begin{equation}\label{eq:statincr}
%   E\bigl[\lVert \XX(s+t) - \XX(s) \rVert^2\bigr] = E\bigl[\lVert \XX(t) - \XX(0) \rVert^2\bigr],
% \end{equation}
% a standard nonparametric estimator for the particle MSD is given by
% \begin{equation}\label{eq:msdemp}
%   \widehat \msd_{\XX}(n \cdot \dt) = \frac{1}{\nd\cdot(N-n+1)} \sum_{i=0}^{N-n} \lVert \XX_{n+i} - \XX_i \rVert^2.
% \end{equation}
% Based on the linear relation
% \begin{equation}\label{eq:logreg}
%   \log \msd_{\XX}(t) = \log 2D + \alpha \log t
% \end{equation}
% over the subdiffusion timescale $t \in (\tmin, \tmax)$, a commonly-used subdiffusion estimator~\citep[e.g.,][]{gal.etal13} is obtained from the  least-squares regression of $y_n = \log \bigl(\widehat \msd_{\XX}(n\cdot \dt)\bigr)$ onto $x_n = \log(n\cdot \dt)$, namely
% \begin{equation}\label{eq:ols}
%   \hat \alpha = \frac{\sum_{n=0}^N(y_n - \bar y)(x_n - \bar x)}{\sum_{n=0}^N(x_n - \bar x)^2}, \qquad \hat D = \tfrac 1 2 \exp(\bar y - \hat \alpha \bar x).
% \end{equation}

\section{Software}

Main idea: efficient likelihoods/posteriors for $\tth = (\mmu, \SSi, \pph)$ under
\begin{equation}
  \label{eq:mnmod}
  \dXX \sim \matnorm(\FF_{\pph} \mmu, \VV_{\pph}, \SSi),
\end{equation}
where $\VV_{\pph} = \toep(\gga_{\pph})$ is a Toeplitz matrix with autocorrelation $\gga_{\pph}$.

% \subsection[modelobject class]{\class{model\_object} class}

% Inputs:
% \begin{itemize}
% \item $\dXX$.
% \item Autocorrelation function \code{acf(phi)} which returns $\gga_{\pph}$ (information about $\dt$ is built in to \code{acf()}).
% \item Drift function \code{drift(phi)} which returns a vector or matrix of drift functions.
% \item Both \code{acf} and \code{drift} can be input as vectors or matrices, in which case it is assumed they do not depend on $\pph$.  Can also be scalars as can be specified in the \pkg{LMN} package.
% \item Optionally, gradient vectors (or matrices or tensors) for \code{acf()} and \code{drift()}.
% \end{itemize}

% Outputs:
% \begin{itemize}
% \item Profile likelihood function \code{nlp(phi)}.
% \item Residual function \code{resid(phi)}.
% \item Likelihood function \code{loglik(phi, mu, Sigma)}.
% \item Conditional MLEs \code{mu_hat(phi)} and \code{Sigma_hat(phi)}.
% \end{itemize}
% Note: can we implement this with S3 classes?

\subsection[modelfit class]{\class{model\_fit} class}

The basic idea is that it's better to tailor the parameter fitting to a specific model, i.e., choice of optimization method, penalty function, etc.  We provide a couple of fitting methods for commonly used models in the package, e.g., \class{fbm\_fit}, \class{farma\_fit}, \class{fsd\_fit}.  However, the output of model fitting should be somewhat generic, so that we can write generic code for downstream analysis, e.g., confidence intervals, p-values, predictions, etc.

Outputs:
\begin{itemize}
\item $\hat \pps$, where $\pps = \ctrans(\tth)$ is reparametrized into a computational basis (see below).
\item $\hat{\var}(\hat \pps)$, provided \code{var_calc = TRUE}.
\item Functions \code{trans(mu, Sigma, phi)} and \code{itrans(psi)} to go back and forth between the two parametrizations.
\item Residuals (if \code{resid_calc = TRUE}).
\item Loglikelihood $\ell(\dXX \mid \tth)$ (for AIC and BIC).
\end{itemize}

\subsection{Computational Basis}

The idea of the computational basis is that the model parameters $\tth = (\mmu, \SSi, \pph)$ are bijectively reparametrized as $\pps = \ctrans(\tth)$, where $\pps$ is a completely unconstrained vector.  A default reparametrization is as follows:
\begin{itemize}
\item $\mmu$ is left as-is.
\item $\SSi$ is transformed to the log-Cholesky scale, i.e., let $\UU$ denote the upper Cholesky factor of $\SSi = \UU'\UU$. The log-Cholesky factor is then defined as the upper-triangular elements of $\UU$ is column-major order, with logs of the diagonal elements.  We will provide functions \fct{log\_chol} and \fct{ilog\_chol} to compute this transformation and its inverse.
\item $\pph$ is transformed as dictated by the specific model.  For example, with fBM we have $\phi = \alpha$, and a natural transformation is $\nu = \logit(\alpha/2)$.  The package will thus also provide generalized \fct{logit} and \fct{ilogit} functions, where generalized means that they are with respect to an arbitrary bounded interval.
\end{itemize}

\section{Illustrations}

\begin{itemize}
\item Empirical methods: MSD and OLS subdiffusion estimators.
\item Parametric estimator: fBM + Static Noise, together with CIs for parameters of interest $(\alpha, D_N)$ (see below).
\item Camera error: ARMA model and compare to fBM estimator.
\item Bad data: filter paths too close together, big jumps, terrible residuals.  Should illustrate how these last two aren't always the same (we've done this somewhere for the JOR data).
\item Heterogeneity: Fitting of normal-mixture model in hierarchical setting.  Largely based on computations from the JASA paper.
\end{itemize}

\subsection{fBM Example}

The \proglang{R} code below shows how to fit the fBM model to all 76 HBE datasets, parallelizing computations with the \proglang{R} package \pkg{parallel}.

FIXME:
\begin{itemize}
\item \code{try-catch} to avoid crashing when optimization doesn't work for a given dataset.
\item Save-to-disk after each fit, so you can resume the computation where you left off.
\item Might be worth using a more complicated model to really show how package works...
\end{itemize}

\subsection{GLE Example}

This is mainly to introduce the GLE and our method to estimate the subdiffusion timescale.

\subsection{Camera Error}

Again, the point is to introduce our work in the AOAS paper.

\subsection{Confidence Intervals}

Suppose that the parametric subdiffusion model parameters are $\pps = \ctrans(\tth) = \ctrans(\mmu, \SSi, \pph)$ in the computational basis. 
% Essentially this means that $\pps$ is a completely unconstrained vector in $\mathbb{R}^{\dim(\tth)}$.
\pkg{subdiff} will take a dataset $\XX$ and return the MLE $\hat \pps$ of $\pps$ and an estimate $\hat \SSi_{\pps}$ of the variance of the MLE, $\var(\hat \pps)$.  Using asymptotic normal theory, the confidence 95\% confidence interval for $\pps_i$ is $\hat \pps_i \pm 1.96 \times \sqrt{[\hat \SSi_{\pps}]_{ii}}$.

Now suppose we want to calculate confidence intervals for the parameters of interest, $\lla = (\alpha, D_N) = \JJ(\pps)$.  We have two ways to do this:
\begin{enumerate}
\item The MLE of $\lla$ is $\hat \lla = \JJ(\hat \pps)$ and its variance estimator is $\hat \SSi_{\lla} = [\frac{\partial\JJ(\hat \pps)}{\partial \lla}] \hat \SSi_{\pps} [\frac{\partial\JJ(\hat \pps)}{\partial \lla}]$.  Construct a symmetric confidence interval using $\pm 1.96$ standard errors exactly as before.
\item While $D_N > 0$, there is nothing preventing $\hat D_N \pm 1.96 \se(\hat D_N)$ to have a negative endpoint, which isn't very useful.  A more accurate CI would first construct a symmetric CI for $\log D_N$ using method 1, then exponentiate the interval to get a CI for $D_N$.  The main downside to this is that the interval is no longer symmetric.
\end{enumerate}

\subsection{Filtering Bad Data}

This will need some work.  Here are the main things we can filter so far:
\begin{itemize}
\item Trajectories that are too short -- presumably because they are/have travelled too far in the $z$ direction, in which case tracking measurements become less accurate.
\item For some experimental data we have a time-dependent signal-to-noise ratio (SNR).  So we can filter trajectories for which the SNR drops below a given level.
\item Another method of filtering is based on the within-path van Hove correlations.  The basic idea is as follows.  Suppose that the trajectory increments $\dXX$ follow the matrix-normal distribution~\eqref{eq:mnmod}, and consider the linear transformation
  \[
    \dZZ_{N\times\nd} = \VV_{\pph}^{-1/2}(\dXX - \FF_{\pph}\mmu)\SSi^{-1/2}.
  \]
  Then the null hypothesis that the model is correct, is equivalent to
  \[
    H_0: \Delta Z_{ij} \iid \N(0,1).
  \]
  This provides an extremely powerful method of testing whether the within-path van Hove correlations are Gaussian via p-values against $H_0$.  These p-values are calculated with the functions \fct{ad\_test} and \fct{sw\_test}.  As for the specific square-roots of $\VV_{\pph}$ and $\SSi$, the recommendation is to pick the Cholesky decomposition for the former, and the eigendecomposition for the latter.  This is implemented in the function \fct{lsc\_resid}.
  
\end{itemize}

\subsection{Heterogeneity modeling}

Let's suppose we wish to identify $K$ different regimes of dynamic motion for particles in a given medium.  A data-generating mechanism for this is:
\begin{equation}
  \label{eq:hetmod}
  \begin{split}
  \dXX_i \mid \tth_i & \ind \matnorm(\FF_{\pph_i}\mmu_i, \VV_{\pph_i}, \SSi_i) \\
  \pps_i = g(\tth_i) & \iid \sum_{k=1}^K \rho_k \N(\lla_k, \OOm_k),
  \end{split}
\end{equation}
where $\dXX_i$ are the increments for particle $i$ and $\pps_i = g(\tth_i)$ is a parameter transformation to a scale on which the particle-wise MLEs $\hat \pps_i$ are approximately normal:
\[
  \hat \pps_i \mid \pps_i \approx \N(\pps_i, \FI(\pps_i)^{-1}),
\]
where $\FI(\pps_i) = -E[\frac{\partial^2}{\partial \pps \partial \pps'} \ell(\pps_i \mid \dXX_i)]$ is the Fisher information.  Upon replacing $\FI(\pps_i)^{-1}$ by an estimator $\hat \SSi_i$, model~\eqref{eq:hetmod} approximately simplifies to 
\[
  \hat \pps_i \mid \pps_i \ind \N(\pps_i, \hat\SSi_i), \qquad \pps_i \iid \sum_{k=1}^K \rho_k \N(\lla_k, \OOm_k),
\]
for which there are fast Bayesian methods to estimate the posterior distribution of the model parameters $p(\rrh, \lla, \OOm, \TTh, \PPs \mid \dXX)$.  The usual approach is to consider the indicator variables $I_i \in \{1, \ldots, K\}$ as missing data.  Also, a useful function for some of the normal conditionals is \fct{rRxNorm} in the \pkg{mniw} package.  For convenience, you should use the same priors as in the \pkg{mniw} package, i.e.,
\begin{align*}
  (\lla_k, \OOm_k) & \ind \mniw(\LLa_k, \UUp_k, \PPs_k, \nnu_k) \\
  \rrh & \sim \dir(\aal).
\end{align*}

%% -----------------------------------------------------------------------------


\end{document}
